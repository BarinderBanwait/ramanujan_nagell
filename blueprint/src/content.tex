% In this file you should put the actual content of the blueprint.
% It will be used both by the web and the print version.
% It should *not* include the \begin{document}
%
% If you want to split the blueprint content into several files then
% the current file can be a simple sequence of \input. Otherwise It
% can start with a \section or \chapter for instance.

\chapter{The Ramanujan--Nagell Theorem}

The Ramanujan--Nagell equation is the Diophantine equation
\[
  x^2 + 7 = 2^n
\]
where $x$ is an integer and $n$ is a natural number. The theorem, conjectured by
Ramanujan and proved by Nagell, states that the only solutions are
$(x, n) \in \{(\pm 1, 3),\; (\pm 3, 4),\; (\pm 5, 5),\; (\pm 11, 7),\; (\pm 181, 15)\}$.

The proof splits into two cases depending on the parity of $n$: the even case uses a
factorization argument over $\mathbb{Z}$, while the odd case requires algebraic number
theory in the ring of integers of $\mathbb{Q}(\sqrt{-7})$.

\section{Setup: the ring of integers of $\mathbb{Q}(\sqrt{-7})$}

We model $\mathbb{Q}(\sqrt{-7})$ as the quadratic algebra $K = \texttt{QuadraticAlgebra}\;\mathbb{Q}\;(-2)\;1$,
where the generator $\omega$ satisfies $\omega^2 = -2 + \omega$, i.e.\
$\omega = (1 + \sqrt{-7})/2$. We write $R = \mathcal{O}_K$ for the ring of integers,
$\theta = \omega \in R$, and $\theta' = 1 - \omega = (1 - \sqrt{-7})/2 \in R$.

\begin{lemma}[Integrality of $\omega$]
  \label{lem:is_integral_omega}
  \lean{is_integral_ω}
  \leanok
  The element $\omega \in K$ is integral over $\mathbb{Z}$: it satisfies $X^2 - X + 2 = 0$.
\end{lemma}

\begin{lemma}[Integrality of $1 - \omega$]
  \label{lem:is_integral_one_sub_omega}
  \lean{is_integral_one_sub_ω}
  \uses{lem:is_integral_omega}
  The element $1 - \omega \in K$ is integral over $\mathbb{Z}$.
\end{lemma}

\begin{lemma}[Minimal polynomial]
  \label{lem:my_minpoly}
  \lean{my_minpoly}
  \uses{lem:is_integral_omega}
  The minimal polynomial of $\theta$ over $\mathbb{Z}$ is $X^2 - X + 2$.
\end{lemma}

\begin{lemma}[Monogenicity]
  \label{lem:span_eq_top}
  \lean{span_eq_top}
  \uses{lem:is_integral_omega}
  The ring of integers $R$ is generated by $\theta$ over $\mathbb{Z}$: $\mathbb{Z}[\theta] = R$.
\end{lemma}

\begin{lemma}[Class number one]
  \label{lem:class_number_one}
  \lean{class_number_one}
  The ring of integers $R$ is a unique factorization domain
  (equivalently, the class number of $\mathbb{Q}(\sqrt{-7})$ is $1$).
\end{lemma}

\begin{lemma}[Units are $\pm 1$]
  \label{lem:units_pm_one}
  \lean{units_pm_one}
  The only units in $R$ are $\pm 1$.
\end{lemma}

\begin{lemma}[Factorization of $2$]
  \label{lem:two_factorisation_R}
  \lean{two_factorisation_R}
  \uses{lem:is_integral_omega}
  In $R$, we have $\theta \cdot (1 - \theta) = 2$, i.e.\
  $\frac{1+\sqrt{-7}}{2} \cdot \frac{1-\sqrt{-7}}{2} = 2$.
\end{lemma}

\section{Parity lemmas}

\begin{lemma}[Odd square implies odd root]
  \label{lem:sq_odd_then_odd}
  \lean{sq_odd_then_odd}
  \leanok
  If $x^2$ is odd, then $x$ is odd.
\end{lemma}

\begin{lemma}[Powers of two are not odd]
  \label{lem:not_odd_two_pow}
  \lean{not_odd_two_pow}
  \leanok
  For $n \geq 1$, the number $2^n$ is not odd.
\end{lemma}

\begin{lemma}[$2^n - 7$ is odd]
  \label{lem:two_pow_min_seven_odd}
  \lean{two_pow_min_seven_odd}
  \leanok
  For all $n \neq 0$, the integer $2^n - 7$ is odd.
\end{lemma}

\begin{lemma}[$x$ is odd]
  \label{lem:x_is_odd}
  \lean{x_is_odd}
  \uses{lem:sq_odd_then_odd, lem:two_pow_min_seven_odd}
  \leanok
  If $x^2 + 7 = 2^n$ with $n \neq 0$, then $x$ is odd.
\end{lemma}

\section{The even case}

When $n$ is even, say $n = 2k$, the equation becomes $x^2 + 7 = 2^{2k}$, which factors
over $\mathbb{Z}$ as $(2^k + x)(2^k - x) = 7$. Since $7$ is prime, this forces
$n = 4$ and $x = \pm 3$.

\begin{lemma}[Factorization over $\mathbb{Z}$]
  \label{lem:my_amazing_thing}
  \lean{my_amazing_thing}
  \leanok
  If $(2^k + x)(2^k - x) = 7$, then either $2^k - x = 1$ and $2^k + x = 7$,
  or $2^k - x = 7$ and $2^k + x = 1$.
\end{lemma}

\section{The odd case}

When $n$ is odd and $n \geq 5$, the proof works in the ring of integers of
$\mathbb{Q}(\sqrt{-7})$. Setting $m = n - 2$, we divide the equation by $4$
to obtain $(x^2 + 7)/4 = 2^m$, which factors in $R$ as
\[
  \frac{x + \sqrt{-7}}{2} \cdot \frac{x - \sqrt{-7}}{2}
  = \theta^m \cdot \theta'^{\,m}.
\]
The division by $4$ is deliberate: it makes the difference of the two factors equal to
$\sqrt{-7} = 2\theta - 1$ (rather than $2\sqrt{-7}$), which simplifies the coprimality argument.

\subsection{Key intermediate result}

The following lemma is the heart of the odd case. Its proof is decomposed into four
exercises (Lemmas~\ref{lem:factors_in_R}--\ref{lem:eliminate_x}).

\begin{lemma}[Main $m$-condition]
  \label{lem:main_m_condition}
  \lean{main_m_condition}
  \uses{lem:factors_in_R, lem:coprime, lem:ufd_association, lem:eliminate_x}
  \leanok
  For all integers $x$ and odd $m \geq 3$, if $(x^2 + 7)/4 = 2^m$, then
  \[
    2\theta - 1 = \theta^m - \theta'^{\,m}
    \quad\text{or}\quad
    -2\theta + 1 = \theta^m - \theta'^{\,m}.
  \]
\end{lemma}

\subsection{Exercises}

The proof of Lemma~\ref{lem:main_m_condition} is structured as a chain of four
lemmas, each sorry'd for now. These are intended as exercises.

\begin{lemma}[Conjugate factors in $R$]
  \label{lem:factors_in_R}
  \lean{factors_in_R_with_product}
  \uses{lem:is_integral_omega, lem:is_integral_one_sub_omega, lem:two_factorisation_R}
  The conjugate factors $(x \pm \sqrt{-7})/2$ lie in $R$ (since $x$ is odd),
  and their product equals $\theta^m \cdot \theta'^{\,m}$.
  Their difference is $2\theta - 1 = \sqrt{-7}$.
\end{lemma}

\begin{lemma}[Coprimality]
  \label{lem:coprime}
  \lean{conjugate_factors_coprime}
  \uses{lem:factors_in_R, lem:two_factorisation_R}
  The conjugate factors are coprime in $R$.
  The only prime factors of $2$ in $R$ are $\theta$ and $\theta'$
  (since $2 = \theta \cdot \theta'$). If either divided both factors,
  it would divide their difference $\sqrt{-7}$, but $N(\sqrt{-7}) = 7$ is
  not divisible by $N(\theta) = N(\theta') = 2$.
\end{lemma}

\begin{lemma}[UFD power association]
  \label{lem:ufd_association}
  \lean{ufd_power_association}
  \uses{lem:coprime, lem:class_number_one, lem:units_pm_one}
  If $\alpha \cdot \beta = \theta^m \cdot \theta'^{\,m}$ and $\gcd(\alpha, \beta) = 1$ in the
  UFD $R$, then $\alpha = \pm\theta^m$ or $\alpha = \pm\theta'^{\,m}$.
  This combines unique factorization (\texttt{class\_number\_one}) with the fact that the
  only units are $\pm 1$ (\texttt{units\_pm\_one}).
\end{lemma}

\begin{lemma}[Eliminate $x$]
  \label{lem:eliminate_x}
  \lean{eliminate_x_conclude}
  \uses{lem:ufd_association}
  From $\alpha = \pm\theta^m$ or $\alpha = \pm\theta'^{\,m}$, use the product relation to
  determine $\beta$, then take the difference $\alpha - \beta = 2\theta - 1$ to
  eliminate $x$ and obtain the conclusion of Lemma~\ref{lem:main_m_condition}.
\end{lemma}

\subsection{From the $m$-condition to finitely many solutions}

\begin{lemma}[Reduction by dividing by $4$]
  \label{lem:reduction_divide_by_4}
  \lean{reduction_divide_by_4}
  \leanok
  If $n$ is odd with $n \geq 5$ and $x^2 + 7 = 2^n$, then $(x^2 + 7)/4 = 2^{n-2}$.
\end{lemma}

\begin{lemma}[Mod $7$ constraint]
  \label{lem:odd_case_mod_seven_constraint}
  \lean{odd_case_mod_seven_constraint}
  \uses{lem:main_m_condition, lem:reduction_divide_by_4}
  If $n$ is odd with $n \geq 5$ and $x^2 + 7 = 2^n$, then
  $(-2)^{n-3} \equiv n - 2 \pmod{7}$.

  This follows from the $m$-condition (Lemma~\ref{lem:main_m_condition}) by expanding
  $\theta^m - \theta'^{\,m}$ via the binomial theorem and reducing modulo $7$.
\end{lemma}

\begin{theorem}[Odd case: only three values]
  \label{thm:odd_case_only_three_values}
  \lean{odd_case_only_three_values}
  \uses{lem:odd_case_mod_seven_constraint}
  If $n$ is odd with $n \geq 5$ and $x^2 + 7 = 2^n$, then $n \in \{5, 7, 15\}$.

  This follows from $(-2)^{n-3} \equiv n - 2 \pmod{7}$ together with
  Fermat's little theorem $2^6 \equiv 1 \pmod{7}$. Checking residues modulo $42$
  and a bounding argument shows $n$ can only be $5$, $7$, or $15$.
\end{theorem}

\section{Main theorem}

\begin{theorem}[Ramanujan--Nagell]
  \label{thm:RamanujanNagell}
  \lean{RamanujanNagell}
  \uses{lem:x_is_odd, lem:my_amazing_thing, thm:odd_case_only_three_values}
  \leanok
  The only integer solutions to $x^2 + 7 = 2^n$ are
  \[
    (x, n) \in \{(\pm 1, 3),\; (\pm 3, 4),\; (\pm 5, 5),\; (\pm 11, 7),\; (\pm 181, 15)\}.
  \]

  \begin{proof}
  First, one shows $n \geq 3$ by bounding $2^n \geq x^2 + 7 \geq 7$.
  Then $x$ must be odd (Lemma~\ref{lem:x_is_odd}).

  \textbf{Case 1: $n$ even.} Write $n = 2k$. Then $(2^k+x)(2^k-x) = 7$.
  By Lemma~\ref{lem:my_amazing_thing}, the only possibility is $2^k = 4$,
  giving $n = 4$ and $x = \pm 3$.

  \textbf{Case 2: $n$ odd, $n = 3$.} Direct computation gives $x^2 = 1$, so $x = \pm 1$.

  \textbf{Case 3: $n$ odd, $n \geq 5$.} By Theorem~\ref{thm:odd_case_only_three_values},
  $n \in \{5, 7, 15\}$, and direct computation gives the remaining solutions.
  \end{proof}
\end{theorem}
